\chapter{Rayos cósmicos y el campo eléctrico atmosférico}
\label{ch:background}

\section{Descubrimiento de los rayos cósmicos}
Para hablar de rayos cósmicos, debemos primero referirnos a la radiación. Su descubrimiento se remonta al 1896 cuando Henri Becquerel estudiaba materiales fosforescentes, donde descubrió que algunos materiales pueden producir ionización naturalmente, fenómeno por el cual es posible ver algunos de ellos en la oscuridad. \\

Lo que ocurre es que algunos elementos son capaces de emitir partículas radialmente con un nivel de energía determinado, uno de los elementos que más logra emitir partículas energéticas es el Radio (Ra) y en honor al mismo, se le llamó a este fenómeno radiación. No fue sino hasta inicios del siglo XX cuando se logró medir por primera vez este fenómeno usando un electroscopio, dispositivo permitía identificar si un cuerpo estaba o no cargado y, adicionalmente determinar su signo.\\


\begin{figure}[h!]
\begin{center}
\includegraphics[width=0.5\textwidth]{Figures/Electroscopio.png}
\caption[Estructura de un espectroscopio]{El electroscopio contiene en su interior 2 láminas delgadas de un conductor, éstas a su vez están conectadas a una varilla conductora con una esfera al final o parte plana. Cuando la esfera interactúa con una superficie cargada, las 2 láminas adquieren la misma carga de la varilla y por lo tanto se repelen, indicando la presencia de una carga. Imagen tomada de \cite{gordon1883physical}}
\label{ELectroscopio}
\end{center}
\end{figure}



El Electroscopio (ver imágen \ref{ELectroscopio}) una vez cargado comenzará un proceso de descarga consecutivo debido a la conductividad del aire. Al cargar intencionalmente un electroscopio y colocarlo cerca de un elemento radioactivo, éste emitirá gran cantidad de partículas subatómicas con alta energía ionizando el medio en que se encuentre y por consiguiente aumentando su conductividad debido al contenido de iones. Por tal razón la carga del Electroscopio se perderá más rápido, la tasa de perdida de carga del electroscopio se asocia con el nivel de radiación y gracias a esto se logró medir el nivel de radiación emitido por un elemento.\medskip

En 1909, Theodor Wulf fue la primera persona en sugerir la idea de medir la radiación alejado de la superficie terrestre y por tal razón mide la radiación a 300 [m] de altura, en la cima de la Torre Eiffel, en París, usando una versión mejorada del electroscopio aislado desarrollado por Wilson en 1900, Theodor esperando observar una disminución en sus mediciones a medida que se alejaba de la fuente de radiación (el suelo), observó que la medida era casi igual que en la superficie, sugiriendo que el excedente debía provenir del espacio exterior.\medskip

Ese mismo año K. Bergwitz con la ayuda de globos aerostáticos mide el nivel de radiación a una altura de 1300 m encontrando un aumento del 24\% con respecto a la medida en la superficie, dos años después en 1911 el físico Austríaco Victor Hess empieza una serie de de vuelos en globos aerostáticos logrando alturas de 5200 m en 1912. Hess atribuye la fuente de radiación proviene del espacio exterior, además que dado a que no observó variaciones entre el día y la noche, concluyó que el Sol no podía ser la fuente primaria.\medskip

Posteriormente Kolhörstor confirmó los resultados de Victor Hess con innumerables vuelos en los cuales alcanzó alturas de hasta 9200 m. En 1927 J. Clay descubre que éste fenómeno varía con la latitud, sugiriendo la hipótesis (luego verificada), que esto se debía al campo magnético del planeta y que por tal razón estos no son en su mayoría fotones sino partículas cargadas. Victor Hess el premio Nobel de Física en 1936, por el descubrimiento de los rayos cósmico.


\section{Rayos cósmicos primarios}

Los rayos cósmicos (CR, por sus siglas en inglés) son partículas de alta energía, que desde su origen en diferentes procesos astrofísicos son sometidas a diversos mecanismos de aceleración a lo largo del universo hasta que finalmente, algunas partículas alcanzan el planeta Tierra.\medskip

El flujo de partículas que alcanza la Tierra en las capas más altas de la atmósfera se llama CR primario y están compuestos por 90\% protones, 9\% núcleos de helio y 1\% de electrones.\medskip

Los CRs son estudiados puesto que su energía y trayectoria está asociada a aspectos físicos de la fuente de origen como por ejemplo su dimensión, densidad de materia y campos magnéticos presentes. Hoy se sabe que la mayoría de ellos se originan fuera del sistema solar, teniendo los solares una energía del orden de GeV. La medición de la energía se da en electronvoltios [eV], que equivale a la energía necesaria para que un electrón cruce un potencial de 1 V.\medskip

Los CRs con energías > 10 GeV pueden ser originados en el sistema solar ya que la energía cinética asociada a la partícula, sólo se debió alcanzar tras recorrer y ser acelerados distancias superiores al tamaño del mismo sistema solar.\medskip
 
Después de más de 100 años de estudio sabemos que los CRs pueden portar energías que pueden variar desde 1 MeV hasta $ 10^{20} $ eV. El la imagen \ref{Diagrama_Rodilla} se puede observar el espectro de CR primarios.\medskip


\begin{figure}[h!]
\begin{center}
\includegraphics[width=0.85\textwidth]{Figures/Diagrama_Rodilla.PNG}
\caption{Espectro de  rayos cósmicos diferencial en función de la energía. Se puede observar también los diferentes rangos de medición de los principales observatorios. La primera y segunda rodilla indican CR de posible origen galáctico y para energías   $>  10^{20} $ eV origen extragaláctico. Tomado de \cite{mollerach2018progress}.}
\label{Diagrama_Rodilla}
\end{center}
\end{figure}

%\section{Métodos de medicion directos de rayos cósmicos}


%Los métodos de detección directos son aquellos experimentos que permiten determinar el flujo, energía y composición química de los CRs, pero debido a que éstos llegan a las capas más altas de la atmósfera sólo pueden ser medidos de 2 formas, usando globos aerostáticos y satélites. Se llaman métodos directo por que permiten medir de forma individual y precisa los CRs que van llegando.\medskip
 

%Gracias a los globos aerostáticos se lograron las primeras mediciones sin embargo su uso es limitado debido a las alturas que puede alcanzar, la masa que puede cargar así mismo como su coste de operación, el personal a bordo y su recuperación, además del tiempo que podrían esta realizado mediciones. Algunos experimentos con globos son muy complejos, algunos pueden llegar a tener un diámetro de 140 [m], albergando un volumen de 1.1 millones de metros cúbicos de gas Helio, cargando experimentos hasta de 3600 [kg] a altitudes cercanas a los 42 [km].\medskip


%Los satélites por otro lado presentan mayores costes, son de construcción compleja y también tiene una carga útil limitada, su tamaño también debe ser limitado debido que deben ser transportados por un cohete con una capacidad fija en volumen y masa. Su tiempo de medición es muchos más largo, pueden duran años en órbita, realmente depende del tiempo de vida con que son construidos, las primeras detecciones de CRs en satélites corresponden a rayos de poca energía, cercanos a 1 [GeV].\medskip

%En ambos casos estamos hablando de detectores donde lo que se quiere es tener la mayor área de cobertura posible y como veremos a continuación los detectores son de gran masa, por tal razón hay un rango límite de energía que puede ser detectando y ronda los 100 [TeV] aproximadamente. \medskip

%Existen diferentes tipos de detectores para estudiar los CRs primarios entre lo que destacan lo de tipo Espectrómetro Magnético (EM), dispositivo de seguimiento dentro de una región con un campo magnético producido por un solenoide, bien sea permanente o de tipo superconductivo, los DTR (Transition Raiation Detector) con los cuales es posible identificar el factor de Lorentz para luego calcular la energía de la partícula incidente, los tipo RICH (The Ring-imaging Cherenkov Detector), usados para estimar la velocidad de la partícula con alta precisión, detectores Cherenkov y los de tipo calorímetros.\medskip

%El método calorímetro es usado para determinar la energía de los CRs. Los calorímetros usando en globos aerostáticos y satélites son los mismos usandos en los aceleradores de partículas, solo que con limitaciones de tamaño y peso, aquí la energía cinética de la partícula incidente es absorbida o disipada a través de la excitación o ionización del material absorbente, lo cual desencadena una cascada de partículas secundarias en el material.\medskip

%Algunos satélites llevan consigo abordo detectores de tipo calorímetros. Éste tipo de detectores deben ser lo suficientemente denso para que la partícula sea absorbida y son de dimensiones limitadas. Algunos pueden llegar a ocupar 1 [$ m_^{2} $] y pesar aproximadamente 1 tonelada lo cual es mucho para un satélite. Algunas misiones pensadas para realizar mediciones en el rango de [GeV] a [Tev] usan espectrómetros magnéticos o detectores Cherenkov. Para partículas más energéticas se usan Detectores de transmisión de Radiación.\medskip

%Los satélites que usan instrumentos sensibles a los CRs, por ejemplo el Interplanetary Monitoring Platform 7 y 8 (IMP) lanzado en 1970  logró mediciones del orden de 100 [MeV], sin embargo este tipo de instrumentos se ve afectado significantemente por el Sol, limitando su tiempo de vida. Otros satélites fueron High-Energy Astrophysics Observatory (HEAO-3), The Cosmic Ray Nuclei ( CRN), el experimento PAMELA, The Payload for Antimatter Exploration and Light-nuclei Astrophysics, permitiendo estudios de hasta cientos de GeV.

\section{Rayos cósmicos secundarios}

Los CRs primarios, al llegar a las capas más altas de la atmósfera interactúan con los núcleos de las moléculas de los gases allí presentes, generando lo que se conoce como una lluvia de partículas secundarias o lluvia aérea extendida (EAS pos su siga en inglés) que a medida que avanza produce más partículas con menos energía, ver figura \ref{Cascada}. \medskip

Durante los años 1938-1939 Pierre Auger y sus colegas realizaron detecciones de rayos cósmicos a nivel del suelo usando detectores separados con una distancia de 200 m identificando coincidencias temporales, dando una primera aproximación a que las partículas en la atmósfera eran realmente partículas secundarias, producidas de una sola partícula primaria.\medskip


Los CRs secundarios están compuestos de fotones, electrones y positrones que constituyen la componente electromagnética (EM), los muones y neutrinos constituyen la componente penetrante, ver imagen \ref{Cascada}, y mesones y bariones que constituyen la componente hadrónica.\medskip

\begin{figure}[h!]
\begin{center}
\includegraphics[width=0.85\textwidth]{Figures/Cascada.PNG}
\caption{Cascada de partículas secundarias tipo EM en la parte izquierda y una generada por núcleos pesados en la parte derecha llamada también cascada hadrónica. En ambos casos se observa una partícula primaria que choca con los núcleos de los átomos de la atmósfera produciendo una reacción en cadena que genera muchas más partículas secundarias. Tomado de \cite{mollerach2018progress}}
\label{Cascada}
\end{center}
\end{figure}

La cascada de partículas secundarias tiene un frente curvo que presenta un ángulo con respecto a la superficie, ver figura \cite{Frente_Cadada}. El frente de las EAS permite identificar el eje de la lluvia y en consecuencia la dirección de procedencia del CR primario que la desencadenó.


con qué ángulo ingresó al planeta, vital para poder estimar la procedencia del CR primario que desencadenó la lluvia en primer lugar. \medskip


\begin{figure}[h!]
\begin{center}
\includegraphics[width=0.85\textwidth]{Figures/Frente.PNG}
\caption{Frente de lluvia de partículas secundarias aproximandose a un grupo de detectores en el suelo. Tomado de \cite{Spurio2015}}
\label{Frente_Cadada}
\end{center}
\end{figure}

\section{Detección de rayos cósmicos secundarios}


Mediante la medición directa de CRs no es posible detectar partículas de muy alta energía $> 10^{15}$ eV, cuyo flujo decae a decenas de partículas por $ m^{2} $ por año.\medskip

Ya que el flujo de CR primario es muy pequeño y el área de los detectores instalados en satélites es de uno pocos metros la detección directa se hace bastante ineficiente. En el método de detección indirecto el detector no mide la partícula primaria, por el contrario, los instrumentos miden la cascada de partículas secundarias que desencadena esta partícula primaria. Dichos observatorios cubren grandes extensiones de superficie ($ > 10 km^{2}$) con el fin de aumentar su sensibilidad a CR de alta energía y de la misma manera el número de eventos registrados.\medskip


Estos arreglos de detectores son construidos a nivel del suelo y son usualmente llamados como experimentos de detección indirecta. Son experimentos de larga duración que permite el estudio del flujo, masa y dirección de los CR primarios de altas energías. Para determinar la dirección de EAS, se usan los datos del tiempo de llegada de las partículas secundarias en los diferentes detectores, mediante de la densidad de partículas secundarias detectadas con la cual se estima la energía del primario y a través de la componente muónica de la EAS se determina la masa del primario usando detectores Cherenkov de agua, detectores de centelleo, detectores de fluorescencia y Cherenkov en el aire (IACT).\medskip

Existen 2 forma de medir y registrar las EAS, se puede clasificar de la siguiente forma; Detectores que miden el contenido de la lluvia de partículas en el suelo; detectores que miden la luz producida por la propagación de la EAS en la atmósfera. Este último tipo con la desventaja de que sólo se pueden tomar datos en noches oscuras, sin Luna.\medskip

Para el estudio de CRs se usan gran variedad de detectores, con el objetivo de poder medir la mayor cantidad de variables posibles y de paso eliminar más variables de los modelos matemáticos con los cuales se comparan los datos de los detectores para su posterior validación. Los detectores se organizan en cuadrículas y separados por distancia que van desde decenas de metros hasta kilómetros. \medskip

%Como bien se sabe esta interacción ocurre en la atmósfera, quien de cierta forma actúa como un calorímetro, reteniendo las partículas más energéticas, que son transformadas en otras partículas detectadas en la superficie, con esa información es posible reconstruir la cascada para poder determinar que tipo de partícula originó la cascada, con qué energía y su lugar de procedencia.\medskip
 

La atmósfera actúa como un calorímetro imperfecto el cual forma parte del sistema de detección, por lo tanto conocer su comportamiento y poder registrarlo en conjunto con lo datos de los detectores es de vital importancia, de aquí la importancia.\medskip


El principal parámetro en lo que respecta al desarrollo de la cascada de partículas secundarias es la cantidad de materia que hay arriba de la atmósfera en donde el CR primario comienza su interacción. Por ello se haba de profundidad atmosférica vertical (ver figura \ref{Profundidad_Atmosferica}), con el cual es posible estimar dichos parámetros para cualquier altura, esto se logra gracias a modelos termodinámicos.\medskip


Finalmente los datos experimentales obtenidos son siempre comparados con los modelos de predicciones de la lluvia de partículas desarrollada en la atmósfera usando simulaciones con el método Monte Carlo.\medskip

Los experimentos de EAS pueden medir flujos de CR en determinados rangos del espectro.

Para RC con energías del orden de $ 10^{15} $ eV la separación óptima entre detectores deberá ser del orden de decenas de metros, para CR de mayor energía la separación deberá ser aumentada. Para los mas energéticos, mayores a $ 10^{18} $ [eV], la separación deberá ser del orden de kilómetros.\medskip


Unos de lo primeros observatorios de CRs fue inaugurado en 1960 por J. Linsley, L. Scarsi and B. Rossi en Volcano Ranch, New Mexico. Contaba con 20 centelladores de  3 [$ m^{2} $] cada uno con separaciones entre ellos de hasta 900 [m] cubriendo una superficie de hasta 8 [$ m^{2} $]. El Extensive Air Shower on Top (EAS-TOP) es un arreglo de detectores ubicado en CampoImper- atore,  Italia, a una altura de 2005 [m], con  una profundidad ammosférica de 820 [$ g/cm^{2} $]. El detector contaba con 35 centelladores de 10 [$ m^{2} $] cada uno separados por 20 [m]  según \cite{Spurio2015}\medskip


Los principales observatorios de rayos cósmicos en Alemania son el The KArlsruhe Shower Core and Array Detecto (KASCADE) dedicado al estudio de CRs con energías entre los $ 10^{14} $ [eV] hasta $ 10^{17} $ [eV] y CASCADE-Grande, en Italia EAS-TOP y en Tibet. Los arreglos de detectores están ubicados normalmente en profundidades atmosféricas de alrededor de 800 [$ g/cm^{2} $] y a nivel del mar. Existen excepciones como el Tibet air-shower array contruido en Yangbajing con una altitud de 4300 msnm con una profundidad atmosférica de 606 [$ g/cm^{2} $] \cite{Spurio2015}. \medskip

El observatorio Pierre Auger ubicado en la ciudad de Malargüe, Argentina es un experimento del cual participan científicos de más de 100 instituciones en 18 países, este observatorio cuenta con 1600 detectores Cherenkov y 24 telescopios de fluorescencia abarcando una extensión de más de 3000 [$k^{2}$].\medskip

\section{Campo Eléctrico Atmosférico}

Una de las características más importantes de los planetas es su atmósfera, de que está compuesta, en que concentraciones y como es su estructura. La atmósfera del planeta tierra compuesta en un 78\% de nitrógeno, 21 \% oxígeno, seguido por concentraciones menores al 1\% de argón, dióxido de carbono, neón, helio y vapor de agua según datos de la NASA \cite{pagina_Nasa}.

%https://nssdc.gsfc.nasa.gov/planetary/factsheet/earthfact.html

La atmósfera se encuentra categorizada en 5 grandes capas; la primera que abarca desde el nivel del suelo hasta los 10 km de altura y se llama la Troposfera, misma capa en la cual tienen lugar los procesos atmosféricos que definen el clima y donde se forman las nubes, luego la Estratosfera donde se encuentra la capa de Ozono, esta capa llega hasta los 50 km de altura \cite{Pagina_UCAR}, ver figura \ref{Capas}.


\begin{figure}[h!]
 \centering
  \subfloat[]{
   \label{Capas}
    \includegraphics[width=0.45\textwidth]{Figures/Capas.jpg}}
  \subfloat[]{
   \label{Ionosfera}
    \includegraphics[width=0.445\textwidth]{Figures/Ionosfera.jpg}}
 \caption{(a) Capas de la atmósfera . R. Russell \cite{Pagina_UCAR}, (b) Ionosfera. Tomado de \cite{Pagina_UCAR}}
 %\label{Ionosfera}
\end{figure}


La Mesosfera llega los 85 km de altura, capa en donde debido a fricción lo meteoritos se queman, donde se encuentran puntos de más baja presión y temperatura a lo largo de toda la atmósfera. La siguiente capa es la Termósfera, capa donde ocurren los procesos de absorción de energía proveniente de rayos X y radiación ultravioleta, es por ello que la temperatura en esta capa aumenta en un rango que va desde los 500 °C hasta los 2000 °C y llega hasta 1000 km de altura. La última capa es la Exósfera, capa que limita con el espacio exterior, donde la densidad de aire es extremadamente baja, considerándola más espacio exterior que atmósfera, se estima que llega a algún punto entre los 100.000 km y 190.000 km de altura \cite{Pagina_UCAR}.\medskip

Adicionalmente existe una zona que abarca desde la Mesosfera hasta la Termosfera llamada Ionosfera, ver figura \ref{Ionosfera}, donde existe suficientes procesos de ionización debido a la radiación ultravioleta del sol, predominando cargas positivas, al punto que esta capa se comporta como un conductor. Por otra parte la superficie del planeta, en condiciones climáticas normales esta cargada principalmente con cargas negativas, ver figura \ref{Capacitor}.



\begin{figure}[h!]
 \centering
  \subfloat[]{
   \label{Capacitor}
    \includegraphics[width=0.45\textwidth]{Figures/Tierra_Cap.jpg}}
  \subfloat[]{
   \label{Ionosfera}
    \includegraphics[width=0.65\textwidth]{Figures/Ionizacion.jpg}}
 \caption{(a) Analogía del planeta tierra modelado como un gran capacitor esférico .Tomado de \cite{Arizona}, (b) La ionización en la atmósfera debido a los rayos cósmicos predomina sobre 1 km de altura y le sigue la ionización debido a elementos radiactivos presentes en la superficie para alturas menores a 1 km de altura. Tomado de \cite{Arizona}}
 %\label{Ionosfera}
\end{figure}


Al existir una abundancia de cargas positivas en la ionosfera y una abundancia de cargas negativas en la superficie del planeta se genera una diferencia de potencial y con ello un campo eléctrico. Dado la forma esférica del planeta Tierra y al ser la ionosfera una capa que rodea toda la superficie se podría ver el planeta como un capacitor esférico concéntrico, el cual bajo condiciones climáticas genera un campo eléctrico perpendicular a la superficie que ronda entre los 100 V/m y 300 V/m, sin embargo cuado condición de tormenta el campo eléctrico aumenta al orden de los 1000 V/m \cite{Arizona}.\medskip

Hay que mencionar que el aire no es un aislante perfecto y por ello, hay un constante flujo de corriente desde la ionosfera hasta el suelo, muy pequeño, pero que en conjunto según \cite{Arizona} con un campo eléctrico de 200 V/m puede llegar a sumar un total de 2000 A descendiendo desde la ionosfera hasta la superficie. Los mecanismos que permiten el flujo de corriente en atmósfera son diferentes a los presentes en un cable pues mientras que en un cable el flujo de corriente se debe al movimiento de electrones libres, en la atmósfera se debe al movimientos de iones presentes en el aire. \medskip

Alguien se podría preguntar, ¿Cómo es que se mantiene esta diferencia de potencial?, más si hay un flujo constante de corriente. Según \cite{Arizona}, al planeta le tomaría alrededor de 10 minutos para entrar en equilibrio, para que toda la carga se estabilice como resultado de estas pequeñas corrientes, pero por qué esto no pasa, ¿Por qué está siempre está cargado?



La respuesta inicial fue los rayos, ya que en la mayoría de los casos las descargas tipo nube-suelo son portadores de carga negativa hacia el suelo, pero se descubrir que con solo los rayos esto no es posible. Posteriormente se pensó que eran las tormentas en general  y hoy en se cree que es todo lo anterior y además las descargas de tipo nube-atmósfera que emiten corrientes desde la nubes hasta partes mas altas de la atmósfera \cite{Arizona}. 









\section{Influencia del campo eléctrico atmosférico en las EAS}