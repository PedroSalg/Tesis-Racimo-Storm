\chapter{Objetivos}
\section{\textbf{Objetivo General}}

Diseñar un prototipo de estación meteorológica autónoma para la medición del campo eléctrico atmosférico en observatorios de astropartículas.

\section{\textbf{Objetivos Específicos}}

\begin{itemize}
    \item Diseñar e implementar un sensor de campo eléctrico rápido (0.1 MHz $–$ 1 MHz) y lento (0.1 Hz $–$ 1 kHz) para tormentas eléctricas.
    
    \item  Implementar en la estación de monitoreo sensores complementarios de temperatura, presión, humedad y lluvia.

    \item  Implementar mediante GPS la sincronización temporal de los eventos registrados.
    \item Desarrollar una interfaz gráfica para la visualización de los datos recolectados.
\end{itemize}
    
    %\item Calibrar el hodoscopio con el fin de obtener una respuesta uniforme en cada uno de sus píxeles y obtener una reconstrucción confiable del flujo de muones.
    
    %\item Calibrar el WCD para obtener el punto óptimo de operación que maximice la discriminación entre  $\mu^{\pm}$ y $e^{\pm}, \gamma$.
    
    %\item Acoplar el hodoscopio y el WCD en un sistema conjunto de adquisición que sea robusto e independiente.
    
%    \item Validar el desempeño del Telescopio de Muones (MuTe) en la reducción del ruido en muografía. 
    
    %\item Instalar sistemas periféricos que suministren información acerca del funcionamiento del detector.
    
    %\item Hacer un análisis detallado de las principales fuentes de ruido en la muongrafía a partir de los datos recolectados.
    
%\end{itemize}