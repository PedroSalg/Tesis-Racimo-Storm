\begin{abstract}
\textbf{Título:} Pasantía de investigación en el grupo de Investigación Geomática, Gestión y Optimización de Sistemas, para realizar la evaluación preliminar de residuos aceitosos y materiales.\hfill \break
\textbf{Autor:} Andrea\hfill \break
\textbf{Palabras Clave:} \hfill \break
\textbf{Descripción:}\hfill \break

Los residuos aceitosos, generados en procesos de exploración, transporte, almacenamiento y refinación de petróleo crudo representan en la actualidad un problema ambiental, debido a su composición de varios hidrocarburos de petróleo (PHC), hidrocarburos aromáticos policíclicos, iones de metales pesados, partículas sólidas y algunas sustancias nocivas. Por esta razón, se han estudiado métodos de tratamiento para procesar y eliminar este tipo de residuos con procesos convencionales biológicos, físicos, métodos químicos, térmicos y combinaciones de estos. Se encontró que los tratamientos pueden variar la eficiencia conforme al tiempo y costo de aplicación, de este modo, el tratamiento más económico no significa el más eficiente. La industria del petróleo invierte cantidades considerables para darle una disposición final a los residuos, por esta razón, con el objetivo de reducir costos de tratamiento, brindar una mejor conectividad entre regiones y reducir el impacto ambiental, se propone darles una utilidad a los residuos en la estabilización de materiales pétreos. Posteriormente, se evaluaron las propiedades físicas de los residuos aceitosos y materiales mediante laboratorios, de acuerdo con las normas establecidas por el INVIAS con el fin de identificar posibles componentes que puedan afectar su aplicación en suelos y tratamientos para vías terciarias.
\end{abstract}